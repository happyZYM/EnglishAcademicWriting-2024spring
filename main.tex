\documentclass{article}

% Language setting
% Replace `english' with e.g. `spanish' to change the document language
\usepackage[english]{babel}

% Set page size and margins
% Replace `letterpaper' with `a4paper' for UK/EU standard size
\usepackage[a4paper,top=2cm,bottom=2cm,left=3cm,right=3cm,marginparwidth=1.75cm]{geometry}

% Useful packages
\usepackage{amsmath}
\usepackage{graphicx}
\usepackage[colorlinks=true, linkcolor=blue,backref=true]{hyperref}
\usepackage[hyperref=true,backref=true,style=ieee, backend=biber,isbn=true]{biblatex}
\addbibresource{refinfo.bib}

\title{Utilizing Generative AI for Personalized Learning in Computer Science Education}
\author{Baihan Deng,Daozheng Xue,Xiaoze Fan,Yumin Zhuang}

\begin{document}
\maketitle

\begin{abstract}
Your abstract.
\end{abstract}

\tableofcontents
\label{toc_contenttable}
\newpage

\section{INTRODUCTION}

Considering GenAI's huge effect in the future, its' very important to discuss how can AI help students appropriately; and those CS students who has been helped by GenAI can take part in GenAI's development in return, so it's also an interesting topic. In previous researches, we find that some studies focus on the practicality of AI-assisted programming and examine professional programmers' perspectives on it, like ``A Large-Scale Survey on the Usability of AI Programming Assistants: Successes and Challenges."\cite{liang-2023-LargeScaleSurveyUsability} and ``Expectation vs. Experience: Evaluating the Usability of Code Generation Tools Powered by Large Language Models"\cite{vaithilingam-2022-ExpectationVsExperience}; some studies focus on the general impact of generative AI on education, like ```Students' use of large language models in engineering education: A case study on technology acceptance, perceptions, efficacy, and detection chances"\cite{bernabei-2023-StudentsUseLarge} and ```With Great Power Comes Great Responsibility!': Student and Instructor Perspectives on the influence of LLMs on Undergraduate Engineering Education"\cite{joshi-2023-GreatPowerComes}; others pay attention to exploring AI-assisted computer education(``Beyond Traditional Teaching: Large Language Models as Simulated Teaching Assistants in Computer Science"\cite{liu-2024-TraditionalTeachingLarge}), exploring the possibility of AI teaching assistants from an educator's perspective(``Bob or Bot: Exploring ChatGPT's Answers to University Computer Science Assessment"\cite{richards-2024-BobBotExploring} and ``ChatGPT in the Classroom: An Analysis of Its Strengths and Weaknesses for Solving Undergraduate Computer Science Questions"\cite{joshi-2024-ChatGPTClassroomAnalysis}) or testing AI in answering CS-related questions, investigating the extent of students' trust in AI(``Trust in Generative AI among Students: An exploratory study"\cite{amoozadeh-2024-TrustGenerativeAI}). In summary, the previous studies pay lots of attention to GenAI's effect on education generally,however,we do not find any study  take research on CS students specifically——a hard-to-begin major's students.As a group whose major is related to AI,it's hard to say whether these students' trust level on GenAI is same with other students'. Also, considering different learning content, it remains unknown that whether general study method(with GenAI) suits CS student. So in this article, we'll discuss how does students and instructors trust GenAI and how to guide CS students to use GenAI to assist their personalized learning, in the hope of giving CS students some advise through our study.


\section{METHODOLOGY}

This study will adopt a mixed-method design, combining both qualitative and quantitative approaches to gather comprehensive insights into the research questions. 
The participants include students majoring in Computer Science and from large university SJTU with different academic levels (from the first to the fourth year of college). Instructors or professors teaching CS courses will also be included in the study.

\subsection{Data Collection Instruments}

The questionnaire survey consists of a series of questions, including open-ended questions and multiple-choice questions, focusing on participants' trust in generative AI, their experiences with using AI in CS learning, and their perceptions of its usefulness in educational contexts.



The survey questionnaire was divided into ... sections.


\begin{itemize}
    \item 
    \item 
    \item 
\end{itemize}



Semi-structured interviews was conducted with a subset of participants to get more information about their experiences, attitudes, and suggestions regarding GenAI in CS education. Specifically, we interviewed two Computer Science major students, one professor, and one teaching assistant. The interview focus on the topics below:

 \begin{itemize}
 \item Perceptions of students using GenAI(eg. ChatGPT, GitHub Copilot) in programming assignments.

\item Challenges or barriers encountered when using GenAI as teaching assistant in Computer Science courses.

\item Suggestions for integrating GenAI into CS curriculum and instructional practices.
\end{itemize}

\section{RESULTS}
\subsection{Students and instructors’ disbelief in GenAI}
\subsubsection{The scale of their trust}
\subsubsection{Why they hold that belief}
\subsection{How to improve their trusty}
\subsubsection{Improve the contents ai generates}
\subsubsection{Improve students' skill of utilizing AI}

% \section{Some examples to get started}

% \subsection{How to create Sections and Subsections}

% Simply use the section and subsection commands, as in this example document! With Overleaf, all the formatting and numbering is handled automatically according to the template you've chosen. If you're using Rich Text mode, you can also create new section and subsections via the buttons in the editor toolbar.

% \subsection{How to include Figures}

% First you have to upload the image file from your computer using the upload link in the file-tree menu. Then use the includegraphics command to include it in your document. Use the figure environment and the caption command to add a number and a caption to your figure. See the code for Figure \ref{fig:frog} in this section for an example.

% Note that your figure will automatically be placed in the most appropriate place for it, given the surrounding text and taking into account other figures or tables that may be close by.

% \begin{figure}
% \centering
% \caption{\label{fig:frog}This frog was uploaded via the file-tree menu.}
% \includegraphics[width=0.25\linewidth]{img/frog.jpg}
% \end{figure}

% \subsection{How to add Tables}

% Use the table and tabular environments for basic tables --- see Table~\ref{tab:widgets}, for example.

% \begin{table}
% \centering
% \caption{\label{tab:widgets}An example table.}
% \begin{tabular}{l|r}
% Item & Quantity \\\hline
% Widgets & 42 \\
% Gadgets & 13
% \end{tabular}
% \end{table}

% \subsection{How to add Comments and Track Changes}

% Comments can be added to your project by highlighting some text and clicking ``Add comment'' in the top right of the editor pane. To view existing comments, click on the Review menu in the toolbar above. To reply to a comment, click on the Reply button in the lower right corner of the comment. You can close the Review pane by clicking its name on the toolbar when you're done reviewing for the time being.

% Track changes are available on all our premium plans, and can be toggled on or off using the option at the top of the Review pane. Track changes allow you to keep track of every change made to the document, along with the person making the change.

% \subsection{How to add Lists}

% You can make lists with automatic numbering \dots

% \begin{enumerate}
% \item Like this,
% \item and like this.
% \end{enumerate}
% \dots or bullet points \dots
% \begin{itemize}
% \item Like this,
% \item and like this.
% \end{itemize}

% \subsection{How to write Mathematics}

% \LaTeX{} is great at typesetting mathematics. Let $X_1, X_2, \ldots, X_n$ be a sequence of independent and identically distributed random variables with $\text{E}[X_i] = \mu$ and $\text{Var}[X_i] = \sigma^2 < \infty$, and let
% \[S_n = \frac{X_1 + X_2 + \cdots + X_n}{n}
%       = \frac{1}{n}\sum_{i}^{n} X_i\]
% denote their mean. Then as $n$ approaches infinity, the random variables $\sqrt{n}(S_n - \mu)$ converge in distribution to a normal $\mathcal{N}(0, \sigma^2)$.


% \subsection{How to change the margins and paper size}

% Usually the template you're using will have the page margins and paper size set correctly for that use-case. For example, if you're using a journal article template provided by the journal publisher, that template will be formatted according to their requirements. In these cases, it's best not to alter the margins directly.

% If however you're using a more general template, such as this one, and would like to alter the margins, a common way to do so is via the geometry package. You can find the geometry package loaded in the preamble at the top of this example file.

% \subsection{How to change the document language and spell check settings}

% Overleaf supports many different languages, including multiple different languages within one document.

% To configure the document language, simply edit the option provided to the babel package in the preamble at the top of this example project.

% To change the spell check language, simply open the Overleaf menu at the top left of the editor window, scroll down to the spell check setting, and adjust accordingly.

% \subsection{How to add Citations and a References List}

% You can simply upload a \verb|.bib| file containing your BibTeX entries, created with a tool such as JabRef. You can then cite entries from it, like this: \cite{amoozadeh-2024-TrustGenerativeAI}. Just remember to specify a bibliography style, as well as the filename of the \verb|.bib|.


\newpage
\appendix
\section{REFERENCES}
\nocite{*}
\printbibliography[heading=none]
\end{document}