\documentclass{article}

% Language setting
% Replace `english' with e.g. `spanish' to change the document language
\usepackage[english]{babel}

% Set page size and margins
% Replace `letterpaper' with `a4paper' for UK/EU standard size
\usepackage[a4paper,top=2cm,bottom=2cm,left=3cm,right=3cm,marginparwidth=1.75cm]{geometry}

% Useful packages
\usepackage{amsmath}
\usepackage{graphicx}
\usepackage[colorlinks=true, linkcolor=blue,backref=true]{hyperref}
\usepackage[hyperref=true,backref=true,style=ieee, backend=biber,isbn=true]{biblatex}
\addbibresource{refinfo.bib}

\title{Utilizing Generative AI for Personalized Learning in Computer Science Education}
\author{Baihan Deng,Daozheng Xue,Xiaoze Fan,Yumin Zhuang}

\begin{document}
\maketitle

\begin{abstract}
Your abstract.
\end{abstract}

\tableofcontents
\label{toc_contenttable}
\newpage

\section{INTRODUCTION}

Considering GenAI's huge effect in the future, it is very important to discuss how AI can help students appropriately; and those CS students who have been helped by GenAI can take part in GenAI's development in return, so it is also an interesting topic. In previous researches, we find that some studies focus on the practicality of AI-assisted programming and examine professional programmers' perspectives on it, like ``A Large-Scale Survey on the Usability of AI Programming Assistants: Successes and Challenges."\cite{liang-2023-LargeScaleSurveyUsability} and ``Expectation vs. Experience: Evaluating the Usability of Code Generation Tools Powered by Large Language Models"\cite{vaithilingam-2022-ExpectationVsExperience}; some studies focus on the general impact of generative AI on education, like ```Students' use of large language models in engineering education: A case study on technology acceptance, perceptions, efficacy, and detection chances"\cite{bernabei-2023-StudentsUseLarge} and ```With Great Power Comes Great Responsibility!': Student and Instructor Perspectives on the influence of LLMs on Undergraduate Engineering Education"\cite{joshi-2023-GreatPowerComes}; others pay attention to exploring AI-assisted computer education(``Beyond Traditional Teaching: Large Language Models as Simulated Teaching Assistants in Computer Science"\cite{liu-2024-TraditionalTeachingLarge}), exploring the possibility of AI teaching assistants from an educator's perspective(``Bob or Bot: Exploring ChatGPT's Answers to University Computer Science Assessment"\cite{richards-2024-BobBotExploring} and ``ChatGPT in the Classroom: An Analysis of Its Strengths and Weaknesses for Solving Undergraduate Computer Science Questions"\cite{joshi-2024-ChatGPTClassroomAnalysis}) or testing AI in answering CS-related questions, investigating the extent of students' trust in AI(``Trust in Generative AI among Students: An exploratory study"\cite{amoozadeh-2024-TrustGenerativeAI}). In summary, the previous studies pay much attention to GenAI's effect on education generally,however,we do not find any study taking research on CS students specifically—a hard-to-begin major's students.As a group whose major is related to AI, it is hard to say whether these students' trust level on GenAI is same with other students'. Also, considering different learning content, it remains unknown whether the general study method(with GenAI) suits the CS student. So in this article, we'll discuss how does students and instructors trust GenAI and how to guide CS students to use GenAI to assist their personalized learning, in the hope of giving CS students some advise through our study.


\section{METHODOLOGY}

The methodology for this study is designed to evaluate the impact of generative AI tools on personalized learning in computer science education. The approach used to collect and analyze data encompasses both quantitative and qualitative methods to ensure a comprehensive understanding of the subject.

\subsection{Survey Design}

A structured questionnaire was developed to capture detailed insights from computer science students on their use and perception of generative AI tools. The questionnaire consists of closed and open-ended questions addressing several aspects of GenAI usage in CS educational settings.

The survey questionnaire was divided into three parts.

\begin{itemize}
    \item The first part consisted of several questions including participants' understanding of GenAI and the use frequency of GenAI in their learning. These questions classify participants into two categories, the namely users and non-users of GenAI.
    \item The second part delves into  how participants use GenAI, like which aspects of their studies participants use generative AI, their level of acceptance towards AI-assisted learning, and whether AI has improved their efficiency.
    \item The third part evaluate the participants' trust level of GenAI. Here, we used three more specific questions corresponding to the three levels of questions which will be answered by GenAI.
    \item The fourth part regrads participants' concern about GenAI, such as data privacy and security issues or negative impact on their skill development.
    \item The final part is an open-ended question, asking participants' other opinions on using GenAI in CS learning.
\end{itemize}

By divide the survey questionnaire into these part, we except to get a comprehensive view of participants' opinions on using GenAI on CS learning. Most of the questions in the questionnaire are multiple-choice questions. For questions related to frequency or degree, four options were provided for measurement. For some questions asking for reasons, aspects, multiple choices are provided, along with an "Other" option for participants to write their own answer in case we didn't consider all possible options. These questions setting enable 






%  \begin{itemize}
%  \item Perceptions of students using GenAI(eg. ChatGPT, GitHub Copilot) in programming assignments.

% \item Challenges or barriers encountered when using GenAI as teaching assistant in Computer Science courses.

% \item Suggestions for integrating GenAI into CS curriculum and instructional practices.
% \end{itemize}

\section{RESULTS}
The survey questionnaire was distributed to computer science students at various levels of study, from undergraduate to graduate programs. A total of 101 valid responses were collected. The results are presented and analyzed below.
\subsection{Students' knowledge about AI}
When asked about their level of understanding of generative AI such as ChatGPT, Claude, and Gemini, 42.57\% of respondents indicated they had a general understanding, while 47.52\% said they were familiar with these tools. Only 9.9\% reported being unfamiliar with generative AI.
\subsection{Students' usage of AI}
The majority of respondents (65.35\%) reported using generative AI frequently in their studies, while 23.76\% used it occasionally. Only 4.95\% reported never using generative AI. The main purposes for using these tools included assisting with programming (68.75\%), writing papers (87.5\%), and solving problems encountered in learning (80.21\%).
\subsection{The role AI plays among students}
When asked about the helpfulness of generative AI in their learning, 45.54\% found it very helpful, and 37.62\% found it somewhat helpful. The main benefits reported were improved learning efficiency (82.18\%), expanded knowledge (49.5\%), enhanced programming skills (49.5\%) and others.
\subsubsection{Specific performance of AI in coding tasks}
Regarding the use of AI programming assistants like Copilot, 25.74\% of respondents reported using them frequently, while 26.73\% used them occasionally. When asked about the impact of these tools on their programming efficiency, 30.19\% reported a significant increase, and 45.28\% reported a moderate increase.

In terms of the impact on code correctness, 64.15\% of respondents believed that AI helped improve the correctness of their code, while 26.42\% reported no noticeable impact. As for the impact on debugging efficiency, 43.56\% reported that AI helped improve their debugging efficiency, while 48.51\% reported no noticeable impact.

\subsection{Students’ level of trust in AI}
\subsubsection{The scale of their trust}
Regarding the correctness of AI's explanations of concepts, 57.42\% of respondents believed in AI's abilities, while 34.65\% were neutral. For AI's reasoning abilities, 49.5\% believed in AI, and 36.63\% were neutral. As for AI's creative abilities, 48.51\% believed in AI, while 35.64\% were neutral.
\subsubsection{Why they hold that belief}
The reasons for students' trust or distrust in AI were not directly addressed in the survey. However, the frequent use of AI tools and the reported benefits suggest that positive experiences with AI contribute to students' trust.
\subsection{Other important information}
\paragraph{}
When asked about concerns regarding the use of generative AI, 37.62\% of respondents were worried about data privacy and security issues, while 9.9\% were not concerned. Additionally, 36.63\% of respondents were unsure if generative AI would negatively impact their skill development or employment prospects, while 50.5\% believed it would not have a negative impact.

\paragraph{}
These results provide valuable insights into computer science students' perceptions and use of generative AI tools in their learning. The high adoption rate and reported benefits suggest that AI is playing an increasingly important role in personalized learning for CS students. However, concerns about data privacy and the potential impact on skill development highlight the need for further research and guidance on the responsible use of AI in education.

\newpage
\appendix
\section{REFERENCES}
\nocite{*}
\printbibliography[heading=none]
\end{document}